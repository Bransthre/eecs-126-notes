\chapter{Introduction and Probability Basics}
\textbf{Probability}, a mathematical method of thinking about uncertainty (which is everything in real life).
However, the quantification of uncertainty comes with some difficulty, and this course serves to teach it.

\section{Introduction to Probability Space}
Probability serves to quantify and model uncertainties. While it makes concrete the chances of some event occurring, it builds a model to represent a world where events are uncertain (which is precisely ours).
The modern probability theory has an axiomatic starting point; they can be listed as follows:

\begin{ln-define}{Probability Space}{}
    A probability space $(\Omega, \mathcal{F}, \P)$ is a triple of:
    \[
        \begin{cases}
            \Omega: &\text{ Sample Space} \\
            \mathcal{F}: &\text{ Family of subsets of $\Omega$, called events} \\
            \P: &\text{ Probability Measure}
        \end{cases}
    \]
\end{ln-define}

Among which, $\mathcal{F}$ is known as a $\sigma$-algebra that contains $\Omega$ itself. \\
\begin{ln-define}{$\sigma$-Algebra}{}
    The term $\sigma$-algebra states that $\mathcal{F}$ is closed under complements and countable unions:
    \begin{align*}
        A \in \mathcal{F} &\implies A^C \in \mathcal{F} \\
        \forall i \in \{1, \dots, m\}, A_i \in \mathcal{F} &\implies \cup_{i}^{} A_i \in \mathcal{F}
    \end{align*}
    By DeMorgan's Law, $\mathcal{F}$ is also closed under countable intersections:
    \[\forall i \in \{1, \dots, m\}, A_i \in \mathcal{F} \implies \cap_{i}^{} A_i \in \mathcal{F}\]
    \textbf{The implication is that intersections and unions of events are still events.}
\end{ln-define}
Meanwhile, $\P$ is a function
\[\P : \mathcal{F} \rightarrow [0, 1]\]
And follows the Kolmogorov Axioms:
\begin{ln-axiom}{Kolmogorov Axioms}{}
    The axioms are:
    \begin{bindenum}
        \item[A1] $\forall A \in \mathcal{F}, \P(A) \geq 0$
        \item[A2] $\P(\Omega) = 1$
        \item[A3] $\sigma$-Additivity: if $A_1, \dots, A_n \in \mathcal{F}$ and are disjoint, then $\P(\cup_{i}^{} A_i) = \sum_{i \geq 1}\P(A_i)$.
    \end{bindenum}
    A counterexample of third axiom would be $\P = \text{Uniform}[0, 1]$, where the probability of any value occurring is essentially $0$, so will the probability of the entire sample space $[0, 1]$, yet such value should be $1$.
\end{ln-axiom}

Let us explore this concept via a demonstration:
\begin{ln-explain}{Coinflip Interpreted as a Probability Space}{}
    The sample space would be the possible occurrences: $\{H, T\}$. \\
    The $\sigma$-algebra ($\mathcal{F}$) would be $2^\Omega = \wp(\Omega)$. \\
    And, the probability of these events can be marked as:
    \[\P(H) = p, \P(T) = 1 - p, \P(\emptyset) = 0, \P({H, T}) = 1\]

    And expanding from these idea, for the probability space of $n$ independent flips,
    \[
        \begin{cases}
            \Omega = {\{H, T\}}^n \\
            \mathcal{F} = 2^\Omega \\
            \P(f_1 f_2 \dots f_n) = p^{|\{i:f_i = H\}|} {(1 - p)}^{|\{i:f_i = T\}|}
        \end{cases}
    \]

    However, we may also interpret with a diferrent sample space. Let,
    \[
        \begin{cases}
            \Omega = \text{set of all configurations of atoms} \\
            A = \text{set of all configurations that land to flip $H$} \\
            B = \text{set of all configurations that land to flip $T$}
        \end{cases}
    \]
    granting a different sigma algebra, but a numerically equivalent resulting probability function.
\end{ln-explain}

\subsection{The Consequences of Axiomatic Startpoints}
\begin{ln-theorem}{Complement Probability}{}
    \textbf{Theorem}: $A \in \mathcal{F} \rightarrow \P(A^C) = 1 - \P(A)$. \\
    \tcblower
    \textbf{Proof}: Because $\mathcal{F}$ is a $\sigma$-algebra, $A \in \mathcal{F} \implies A^C \in \mathcal{F}$.
    We acknowledge that $\Omega = A \sqcup A^C$. \\
    Then,
    \[1 \underset{A2}{=} \P(\Omega) = \P(A \sqcup A^C) \underset{A3}{=} \P(A) + \P(A^C)\]
\end{ln-theorem}
\begin{ln-theorem}{Probability of Smaller Event Space}{}
    \textbf{Theorem}: $A, B \in \mathcal{F} \implies \P(A) \leq \P(B)$. \\
    \tcblower
    \textbf{Proof}: Recognizing that $B = A \sqcup (B \backslash A)$, we may conclude that:
    \[\P(B) = \P(A \sqcup (B \backslash A)) \underset{A3}{=} \P(A) + \P(B \backslash A) \underset{A1}{\geq} \P(A)\]
\end{ln-theorem}
\begin{ln-theorem}{Principle of Inclusion-Exclusion}{}
    \textbf{Theorem}: $\P(A \cup B) = \P(A) + \P(B) - \P(A \cap B)$. \\
    \tcblower
    \textbf{Proof}: We may use a Venn Diagram to decompose for this probability and find the exact result.
    Essentially, the ongoing subtractions and additions are to subtract overcounted regions on the venn diagram and add back the oversubtracted regions. \\
    Or, mathematically:
    \[A \cup B = B \sqcup (A \backslash B)\]
    \[\P(A \cup B) \underset{A3}{=} \P(B) + \P(A \backslash (A \cap B)) \underset{A3}{=} \P(B) + \P(A) - \P(A \cap B)\]
\end{ln-theorem}
\begin{ln-theorem}{A Discrete Probability Space}{}
    \textbf{Discrete Probability Space}: For some countable $\Omega$, $\mathcal{F} = 2^\Omega$, where all $\omega \in \Omega$ are events.
    In that case, $\sum_{\omega \in \Omega} \P(\omega) = 1$ (this is otherwise known as the description of a discrete probability space).
\end{ln-theorem}
\begin{ln-theorem}{The Law of Total Probability}{}
    \textbf{Theorem}: If $A_1, \dots \in \mathcal{F}$ partition $\Omega$ (such that $\Omega = \underset{i \geq 1}{\sqcup} A_i)$, then
    \[\forall B \in \mathcal{F}, \P(B) = \sum_{i \geq 1} \P(B \cap A_i)\]
    which can be supported by or support that:
    \[B = \underset{i \geq 1}{\sqcup} (B \cap A_i)\]
\end{ln-theorem}

\section{Rest of the Lecture}
From a mathematician's perspective (or probabilist's), probability spaces are abstract objects where given its triplet components $(\Omega, \mathcal{F}, \P)$, we attempt to perform and inquire about experiments. \\
Meanwhile, a statistician asks how to choose a good model, not how a model works. \\
And last but not least, an engineering scientist wonders how do we select a model that captures the essence of a real-world problem, and how can it draw insights to make quantitative guarnatees and how to control a system effectively.
